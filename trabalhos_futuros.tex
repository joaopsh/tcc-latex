\chapter{Trabalhos Futuros}
A criação das funcionalidades de alterações de informações da conta do usuário e lista de contatos. assim como a validação dos componentes da interface em outros navegadores, como o FireFox, Opera, Safari, Internet Explorer e Microsoft Edge, podem ser implementações rápidas, devido a toda a infraestrutura já estar pronta, assim como a interface, e tornar o sistema útil para algum fim específico que possa se beneficiar das características de comunicação do sistema.

Pensando na utilização por diversos usuários do sistema desenvolvido, existem implementações que devem ser refeitas devido a problemas aparentes de desempenho que a aplicação pode ter, principalmente nas implementações de interação com o mapa, pois a aplicação atual marca todos os usuários \textit{online} no sistema no mapa de cada usuário, o que não é o ideal. Melhorias nesse sentido podem envolver uma pesquisa mais aprofundada nas APIs do Google Maps, para entender melhor as formas de utilização, assim como citamos em trabalhos relacionados, o qual o Uber pode utilizar ferramentas diferentes diferentes para a elaboração do seu mapa. 

Outro ponto importante que envolve este problema de escalabilidade com o sistema de localização, pode ser a utilização das funcionalidades de geolocalização do banco de dados Redis, onde as informações ao invés de serem distribuídas uniformemente para todos os usuários, elas podem ser baseadas na localização do usuário requerente, onde por exemplo, os usuários obtenham informações apenas de outros usuários próximos a ele, fazendo-se cálculos para retorno de dados baseando-se em um determinado raio de distância.

A simulação de ambientes distribuídos e testes de estresse podem ser realizados na aplicação e trazerem resultados interessantes, que podem demonstrar a eficiência da arquitetura desenvolvida para a aplicação. Isto também implicará em novas implementações, como \textit{sticky sessions} e \textit{load balancer}, desmistificando muitos conceitos envolvidos em aplicações que suportam milhares de usuários simultâneos.