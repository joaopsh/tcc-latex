\chapter{Hardware e Software Utilizados}

\section{Codificação e Testes}
\subsection{Ambiente de Desenvolvimento}
O ambiente para desenvolvimento da solução foi criado em um único computador, de caráter pessoal, do tipo notebook, com as seguintes especificações técnicas \cite{notebook-info}:
	\begin{itemize}
		\item Processador Intel Core i5 3230M - 2.60 GHz (T-Max: 3.20 GHz) - 64 Bits - Smart Cache 3 MB L3
	\end{itemize}
	\begin{itemize}
		\item Chipset Intel ® HM75 Express
	\end{itemize}
	\begin{itemize}
		\item Memória RAM 8 GBs DDR3 SDRAM - 1600 MHz
	\end{itemize}
	\begin{itemize}
		\item Disco rígido 500GB - 5.400 RPM
	\end{itemize}
	\begin{itemize}
		\item Rede 10/100 e Wireless 802.11 b/g/n - Link de conexão com a internet de 15 Mbps
	\end{itemize}

Especificamos apenas algumas informações que julgamos serem de relevância para alguma análise de desempenho da aplicação por parte do leitor. Em nossa solução não iremos realizar levantamentos de dados sobre desempenho, pois este não pode ser considerado um ambiente que simule a realidade em servidores de aplicações, do qual seria de relevância este tipo de medição.

O sistema operacional escolhido para nosso ambiente de desenvolvimento foi o CentOS 7. O CentOS é uma distribuição Linux baseada no Red Hat. A Red Hat é uma empresa privada, que possui um O.S. chamado Red Hat Enterprise Linux (RHEL), e tem como proposta oferecer produtos \textit{enterprise} baseados no GNU/Linux. Tendo em conta que toda distribuição baseada no \textit{core} do Linux deve seguir uma série de regras contidas em sua licença, a Red Hat concentra seus faturamentos em suporte as distribuições que ela construiu. Ter um sistema compatível com o RHEL traz alguns benefícios que ainda não foram alcançados sem que se tenha um time de pessoas dedicadas ao desenvolvimento de um \textit{software} específico, como no caso da maioria dos sistemas open-sources, que é o alto padrão em testes contra erros e problemas de segurança, \textit{releases} mais curtos, correções de problemas em curto prazo, entre outros benefícios que o CentOS consegue absorver da comunidade Red Hat. Além disso, o CentOS já possui uma considerável comunidade ativa, que facilita na resoluções de problemas mais específicos e no desenvolvimento de novas funcionalidades. O sistema também já possui grandes \textit{cases} de sucesso, como Facebook \cite{facebook-distro}, Google \cite{google-redhat}, entre outros, o que passa uma maior confiança para a adoção em outros negócios e uma perspectiva de crescimento contínua.

\subsection{Editores de Texto}
Para o desenvolvimento majoritário do \textit{frontend}, utilizamos o editor de texto SublimeText 3 \cite{sublime}, apoiado de alguns \textit{plugins} para auxílio de escrita e design, como Emmit, ColorHighlighter, AlignTab, JSHint, MaterialTheme entre outros, todos podendo ser encontrado no repositório de pacotes packagecontrol.io \cite{packagecontrolio}.

Para o desenvolvimento do \textit{backend} e parte do \textit{frontend}, utilizamos o editor de texto Visual Studio Code 1.6 \cite{vscode}. A escolha por este outro editor de texto se fez principalmente pela funcionalidade de \textit{debug}, que apresentou melhores resultados em sua utilização, além de ser uma funcionalidade nativa da aplicação, assim como a ferramenta para versionamento de código. Também foi utilizado \textit{plugins} semelhantes aos escolhidos no SublimeText para auxílio de escrita de código, todos encontrados no repositório de pacotes padrão do editor.

\section{Back-end}
\subsection{NodeJS}


\subsection{NPM}


\subsection{ExpressJS}


\subsection{SocketIO}


\subsection{MongoDB}


\subsection{Redis}


\section{Front-end}

\subsection{Bower}


\subsection{AngularJS}


\subsection{Angular Material}


\subsection{Gulp}


\subsection{SASS}


\subsection{Webpack}


\subsection{Babel}


\subsection{Nginx}