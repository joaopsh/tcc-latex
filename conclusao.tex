\chapter{Conclusão}
A elaboração deste projeto trabalhou diversas áreas do desenvolvimento de \textit{softwares}, desde a infraestrutura à criação de interfaces. A aplicação de uma tecnologia que proporciona interação em tempo real, foco do nosso tema, foi facilmente introduzida devido as características que as tecnologias escolhidas já propunham, tendo em vista que a \textit{stack} MEAN (MongoDB, ExpressJS, AngularJS, NodeJS) é muito utilizada para a construção de aplicações tempo-real, possibilitando uma pesquisa muito mais fácil devido as diversas soluções semelhantes já elaboradas por outros desenvolvedores.

A exploração deste tema nos proporcionou conhecer melhor muitas das aplicações deste tipo de tecnologia, e principalmente a experiência em se desenvolver uma aplicação deste tipo, tendo em vista a solução que este tipo de tecnologia pode oferecer para diversos problemas tecnológicos. Os trabalhos relacionados apresentam grandes ferramentas que utilizam uma ou mais tecnologias citadas e demonstram a relevância do tema para o tempo atual da tecnologia, levando a trabalhos cada vez maiores e que exploram as diversas complexidades que sistemas tempo-real podem alcançar para atender as necessidades demandadas. Acreditamos que os resultados alcançados atingiram as metas propostas pelo trabalho, e deixam abertas oportunidades para pesquisas mais específicas da tecnologia.

A ideia deste trabalho foi ser como um passo inicial para o entendimento do ambiente tempo-real em aplicações, consequentemente não aplicamos e nem nos aprofundamos em algumas boas práticas, como o desenvolvimento orientado a testes, o que tornou a utilidade da aplicação meramente demonstrativa, mas que com mais trabalho pode se tornar algo utilizável para outros futuros estudos na área em que abrange.



