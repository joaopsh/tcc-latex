\chapter{Conclusão}
A elaboração deste projeto trabalhou diversas áreas do desenvolvimento de \textit{softwares}, desde a infraestrutura à criação de interfaces. A aplicação de uma tecnologia que proporciona interação em tempo real, foco do nosso tema, foi facilmente introduzida devido as características que as tecnologias escolhidas já propunham, tendo em vista que a \textit{stack} MEAN (MongoDB, ExpressJS, AngularJS, NodeJS) é muito utilizada para a construção de aplicações de tempo real, possibilitando uma pesquisa muito mais fácil devido as diversas soluções semelhantes já elaboradas por outros desenvolvedores.

A exploração deste tema nos proporcionou conhecer algumas das aplicações deste tipo de tecnologia, e principalmente a experiência em se desenvolver uma aplicação deste tipo, tendo em vista a solução que este tipo de tecnologia pode oferecer para diversos problemas tecnológicos. Este trabalho pode contribuir para o entendimento de como utilizar um conjunto de tecnologias para o desenvolvimento de uma aplicação de tempo real, que foi estudado o \textit{chat} como caso de uso. Existem diversas ferramentas de sucesso no mercado que utilizam uma ou mais tecnologias citadas, e demonstram a relevância do tema para o tempo atual da tecnologia, levando a trabalhos cada vez maiores e que exploram as diversas complexidades que sistemas de tempo real podem alcançar para atender as necessidades demandadas. Acreditamos que os resultados alcançados atingiram as metas propostas pelo trabalho, e deixam abertas oportunidades para pesquisas mais específicas da tecnologia.

A ideia deste trabalho também foi ser como um passo inicial para o entendimento do ambiente de tempo real em aplicações, consequentemente não aplicamos e nem nos aprofundamos em algumas boas práticas, como o desenvolvimento orientado a testes, o que tornou a utilidade da aplicação meramente demonstrativa, mas que com mais trabalho pode se tornar algo utilizável para outros futuros estudos na área em que abrange.

\section{Trabalhos Futuros}
Para trabalhos futuros a criação das funcionalidades de alterações de informações da conta do usuário e lista de contatos. assim como a validação dos componentes da interface em outros navegadores, como o FireFox, Opera, Safari, Internet Explorer e Microsoft Edge, podem ser implementações rápidas, devido a toda a infraestrutura já estar pronta, assim como a interface, e tornar o sistema útil para algum fim específico que possa se beneficiar das características de comunicação do sistema.

Pensando na utilização por diversos usuários do sistema desenvolvido, existem implementações que devem ser refeitas devido a problemas aparentes de desempenho que a aplicação pode ter, principalmente nas implementações de interação com o mapa, pois a aplicação atual marca todos os usuários \textit{online} no sistema no mapa de cada usuário, o que não é o ideal, pois no caso de muitos usuários online, a marcação de muitos pontos no mapa poderá causar lentidão ou até paralisação total da página. Melhorias nesse sentido podem envolver uma pesquisa mais aprofundada nas APIs do Google Maps, para entender melhor as formas de utilização, assim como citamos em trabalhos relacionados, o qual o Uber pode utilizar ferramentas diferentes diferentes para a elaboração do seu mapa. 

Outro ponto importante que envolve este problema de escalabilidade com o sistema de localização, pode ser a utilização das funcionalidades de geolocalização do banco de dados Redis, onde as informações ao invés de serem distribuídas uniformemente para todos os usuários, elas podem ser baseadas na localização do usuário requerente, onde por exemplo, os usuários obtenham informações apenas de outros usuários próximos a ele, fazendo-se cálculos para retorno de dados baseando-se em um determinado raio de distância.

A simulação de ambientes distribuídos e testes de estresse podem ser realizados na aplicação para demonstrar a eficiência da arquitetura desenvolvida para a aplicação. Isto também implicará em novas implementações, como \textit{sticky sessions} e \textit{load balancer}, mostrando muitos conceitos envolvidos em aplicações que suportam milhares de usuários simultâneos.
