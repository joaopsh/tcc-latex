\documentclass[brazil,ruledheader]{abntifes}
%\documentclass[brazil,twoside,ruledheader]{abntifes}
\usepackage[T1]{fontenc} 
%\usepackage[latin1]{inputenc}
\usepackage[utf8]{inputenc}
\usepackage[brazil]{babel}
\usepackage[]{algorithm2e}
\usepackage{pslatex}
\usepackage{url}
\usepackage{fancyhdr}
\usepackage{graphicx}
\usepackage{amsmath, amsthm, amssymb}
\usepackage{exercise}
\usepackage{makeidx}
\usepackage{setspace}
\usepackage{multicol}    
\usepackage{upquote}
\usepackage{graphicx}
\usepackage{float}
\usepackage{epigraph}

\usepackage{listings}

\lstset{numbers=left, stepnumber=5, firstnumber=1, numberstyle=\tiny, extendedchars=true, breaklines=true, frame=tb, basicstyle=\footnotesize, stringstyle=\ttfamily, showstringspaces=false }

%\makenomenclature

% Para listar programas em C# 
\lstdefinelanguage{cs}
{morekeywords={abstract,event,new,struct,as,explicit,null,switch
		base,extern,object,this,bool,false,operator,throw,
		break,finally,out,true,byte,fixed,override,try,
		case,float,params,typeof,catch,for,private,uint,
		char,foreach,protected,ulong,checked,goto,public,unchecked,
		class,if,readonly,unsafe,const,implicit,ref,ushort,
		continue,in,return,using,decimal,int,sbyte,virtual,
		default,interface,sealed,volatile,delegate,internal,short,void,
		do,is,sizeof,while,double,lock,stackalloc,
		else,long,static,enum,namespace,string, },
	sensitive=false,
	morecomment=[l]{//},
	morecomment=[s]{/*}{*/},
	morestring=[b]",
}

\newcommand{\AUTOR}{João Pedro Souza Homem}
\newcommand{\SEGUNDOAUTOR}{}
\newcommand{\ORIENTADOR}{Marlan Kulberg}
\newcommand{\COORIENTADOR}{}
\newcommand{\TITULO}{Aplicação tempo-real com NodeJS - Chat por localização geográfica}
\newcommand{\CURSO}{Tecnólogo em Tecnologia da Informação e Comunicação}
\newcommand{\GRAU}{Tecnólogo em Tecnologia da Informação e Comunicação}
% \newcommand{\GRAU}{Tecnólogo em Análise e Desenvolvimento de Sistemas}
\newcommand{\INSTITUICAO}{Faculdade de Educação Tecnológica do Estado do Rio de Janeiro Faeterj/Petrópolis}
\newcommand{\ANO}{Dezembro, 2016}
\newcommand{\DATA}{12 de Dezembro de 2016}
\newcommand{\LOCAL}{Petrópolis - RJ}
\newcommand{\epigrafe}{\vspace{1cm}{\raggedright\par\sffamily\slshape\par}}
\begin{document}
	
	\autor{\AUTOR}
	\titulo{\TITULO}
	\orientador{\ORIENTADOR}
	\coorientador{\COORIENTADOR}
	
	\comentario{Trabalho de Conclusão de Curso apresentado à Coordenadoria do Curso de \CURSO\
		da \INSTITUICAO , como requisito parcial para obtenção do título de \GRAU .}
	
	\instituicao{\INSTITUICAO}
	\curso{\CURSO}
	\governo{Governo do Estado do Rio de Janeiro}
	\secretaria{Secretaria de Estado de Ciência e Tecnologia}
	\fundacao{Fundação de Apoio à Escola Técnica}
	\cpti{Centro de Educação Profissional em Tecnologia da Informação}
	\local{\LOCAL}
	\data{\ANO}
	
	\capa
	
	\folhaderosto
	
	% Ficha Catalográfica
	%\begin{figure}
	%\includegraphics[width=11cm]{FichaCatalografica.pdf}
	%\end{figure}
	
	% Folha de Aprovação
	\newpage
	\vfill 
	\null
	\begin{center}
		{\Huge {\bfseries\itshape Folha de Aprovação}}\\[3cm]
		\begin{espacoduplo}
			Trabalho de Conclusão de Curso sob o título \textit{``\TITULO''},
			defendida por \AUTOR\ e aprovada em \DATA, em \LOCAL, pela banca examinadora constituída pelos
			professores: \setlength{\ABNTsignthickness}{0.4pt}
		\end{espacoduplo}
		\setlength{\ABNTsignthickness}{0.4pt}
		
		% ou inserir a página assinada e escaneada aqui
		%\begin{figure}
		%\includegraphics[]{Fo	lhaAprovacao.pdf}
		%\end{figure}
		
		
		\assinatura{Prof. \ORIENTADOR\\ Orientador} 
		\assinatura{Prof. Banca Interna \\ \INSTITUICAO} 
		\assinatura{Prof. Banca Interna \\ \INSTITUICAO} 
		%\assinatura{Prof. Banca Externa \\ Instituto do membro externo}
		
	\end{center}
	
	
	% Folha do Termo de Compromisso
	\newpage
	
	\vfill 
	\null
	\begin{center}
		{\Huge {\bfseries\itshape Declaração de Autor}}\\[3cm]
		\begin{espacoduplo}
			Declaro, para fins de pesquisa acadêmica, didática e técnico-científica, que o presente Trabalho de Conclusão
			de Curso pode ser parcial ou totalmente utilizado desde que se faça referência à fonte e aos autores.
		\end{espacoduplo}
		\setlength{\ABNTsignthickness}{0.4pt}
		\assinatura{\AUTOR}
		Petrópolis, em \DATA
	\end{center}
	
	\chapter*{Agradecimentos}
	Agradeço a todos que de forma direta ou indireta contribuíram para a elaboração deste trabalho através do investimento em minha capacitação, compartilhamento de conhecimento, disponibilização de ferramentas e componentes para desenvolvimento de \textit{softwares}, entre outros não menos importantes.
	
	Dentre todos os relacionados a este agradecimento, gostaria especialmente de agradecer a toda comunidade \textit{open-source}, da qual pude usufruir de diversos recursos que viabilizaram a implementação da aplicação abordada neste documento. A iniciativa \textit{open-source} abre muitos caminhos para a inovação e inclusão na área de desenvolvimento de \textit{softwares} e deve ser incentivada e reconhecida, atribuindo-se os devidos méritos a cada um que adere a iniciativa em seus projetos.
	\vfill 
	\null
	
	\begin{center}
		{\Huge {\bfseries\itshape Epígrafe}}\\[3cm]
		\vspace{15cm}
	\end{center}
	
	\begin{espacoduplo}
	\end{espacoduplo}
	
	% Epigrafe
	\epigraph{"Faça-se simples com ideias complicadas"}{(JoPaseho)}
	
	% Resumo
	\begin{resumo}
		A evolução da tecnologia vem trazendo para as nossas vida diversas inovações que estão mudando a forma de lidar com inúmeras situações. Alcançamos novos patamares em como tratamos os negócios, relações pessoais, a busca por novos conhecimentos, entre outras coisas. Tudo isso nos levou a buscar entender um tipo de aplicação que tem se tornado cada vez mais presente nas ferramentas que utilizamos.
		
		As aplicações tempo-real são o propósito da realização deste trabalho. Primeiramente, buscamos no mercado algumas referências de aplicações para embasar o desenvolvimento de uma aplicação de \textit{chat} por localização geográfica. Encontramos em nossas referências diversos aspectos que foram considerados para a escolha de como e 
		quais tecnologias usar para construir as funcionalidades.
		
		Tendo como base todas as possibilidades que tínhamos para implementação, delimitamos o problema em construir uma ferramenta com tecnologias modernas e utilizadas no mercado por aplicação com fins semelhantes ao nosso, e que demonstrasse o funcionamento elemental da transmissão de dados em tempo real na \textit{web}.
		
		Sendo assim, subdividimos o desenvolvimento em duas partes principais, que são a aplicação servidor e cliente. 
		
		\begin{itemize}
			\item A aplicação servidor desempenhará o papel de receber as informações da aplicação cliente, processá-las, e distribuí-las para os outros clientes através de uma conexão bidirecional, que nos permite implementar uma comunicação tempo-real entre os diversos usuários da aplicação.
		\end{itemize}
		
		\begin{itemize}
			\item  A aplicação cliente, resumidamente pode ser considerada a interface com a qual os usuários interagem com a aplicação. Através dela o usuário poderá encontrar, por de um mapa terrestre, um outro usuário que esteja utilizando a aplicação no mesmo momento, que será identificado como um ponto marcado no local em que os dados de geolocalização coletados pelo servidor informam. Após a localização do usuário, este poderá ser contactado através de uma janela de comunicação por texto.
		\end{itemize}
		
		A arquitetura utilizada foi inspirada em alguns padrões que moldam a infraestrutura da aplicação para privilegiar ações de escalabilidade, portanto apresentamos parte desse esquema implementado em nosso sistema, e uma outra parte complementar de forma teórica.
		
		\begin{description}
			\item[Palavras-chave] Aplicação tempo-real. Chat por localização. Escalabilidade. NodeJS. SocketIO. Redis. MongoDB. AngularJS. Google Maps.
		\end{description}
		
	\end{resumo}
	
	% Abstract
	\begin{abstract}
		The evolution of technology has brought to our lives several innovations that are changing the way to deal with many situations. We reached new heights in how we treat business, personal relationships, the search for new knowledge, among other things. All this led us to seek to understand a type of application that has become increasingly present in the tools we use.
		
		The real-time applications are the subject of this work. First of all, we seek some applications's references to support us with the development of a geographic location based chat. We found in our references several aspects that were considered for the choice of how and which technologies to use to build the functionalities.
		
		Based on all the possibilities that we had to implement, we delimited the problem of building a tool with modern technology and used in the market for application purposes similar to ours, and to demonstrate the elemental operation of real-time data transmission in the web.
		
		Thus, the development was subdivided into two main parts, which are the application server and client.
		
		\begin {itemize}
		\item The application server will play the role of receiving the client application information, process them, and distribute them to other clients through a bidirectional connection that allows us to implement a real-time communication between the application's users.
		\end {itemize}
		
		\begin {itemize}
		\item The client application can be briefly considered the interface which users interact with the application. Through it, you can find, by a terrestrial map, another user who is using the application at the same time, it will be identified as a marked point where the geolocation data collected by the server report. After found the user location, it can be contacted by a text communication window.
		\end {itemize}
		
		The architecture used was inspired by some standards that shape the application infrastructure to favor scalability actions, therefore we present part of this scheme implemented in our system, and another complementary part theoretically.
		
		\begin{description}
			\item[Keywords] Real-time Application. Chat by location. Scalability. NodeJS. SocketIO. Redis. MongoDB. AngularJS. Google Maps.
		\end{description}
		
	\end{abstract}
	\listoffigures
	
	\listoftables
	
	%Lista de abreviaturas
	\newpage
	
	\vfill 
	\null
	\begin{center}
		{\Huge {\bfseries\itshape Lista de Abreviaturas}}\\[3cm]
	\end{center}
	
	\begin{espacoduplo}
		\begin{description}
			\item[PUB/SUB] Publish/Subscription
			\item[SQL] Structured Query Language
			\item[NoSQL] Not Only Structured Query Language
			\item[SPA] Single Page Application
			\item[MVVM] Model-View-ViewModel
			\item[REST] Representational State Transfer
			\item[JSON] JavaScript Object Notation
			\item[API] Application Programming Interface
			\item[I/O] Input/Output
			\item[O.S.] Operation System
			\item[RHEL] Red Hat Enterprise Linux
		\end{description}
	\end{espacoduplo}
	
	\tableofcontents{}
	
	% Capítulo
\chapter{Introdução}
Os avanços da tecnologia em conjunto com a internet têm ditado aos seus usuários novas formas de consumo de dados e informações. Além disto, um assunto tem ganhado cada vez mais destaque no provimento de dados na rede: a periodicidade. A inclusão dos meios tecnológicos no cotidiano das pessoas e o grande aumento do volume de dados gerados tem desenvolvido uma nova tendência de informações instantâneas, o que nos leva a elaborar meios para atender a esta nova demanda.

Alguns questionamentos aos modelos atuais de consumo de dados, que apesar de muito bem sucedido em diversas áreas, surgem como uma necessidade ainda maior pelo consumo de informações. O \textit{e-mail}, tradicional meio de comunicação entre pessoas, oferecendo uma boa razão de confiabilidade e formalidade, já é questionado por \textit{startups} \cite{startup-email-innovation1} \cite{startup-email-innovation2} que acreditam no futuro da \textit{web} em tempo real.

% Seção
\section{Motivação} 
As aplicações tempo-real estão surgindo como requisito de muitos dos novos sistemas em desenvolvimento, e buscam atender consumidores que já possuem em seus negócios a necessidade da informação imediata. Neste trabalho iremos desenvolver uma aplicação que explora de forma bem específica essa necessidade da troca de informações entre usuário e aplicação em tempo real.

A aplicação desenvolvida será um \textit{web chat} de comunicação por texto baseado em localização geográfica. As duas principais características do projeto, comunicação por texto e localização geográfica, traz para si aspectos modernos, que exigem a utilização de ferramentas e técnicas atuais de desenvolvimento para a \textit{web}, tendo em vista que deveremos levar em consideração diversos conceitos, como compatibilidade, escalabilidade, infraestrutura de sistemas, entre outros. Tudo isto serve como motivação para atualização e aprendizado de diversos tópicos relacionados ao desenvolvimento de \textit{softwares}.   

% Seção
\section{Objetivos}
A finalidade do projeto será demonstrar como pode ser feita a transferência de dados em tempo real entre servidores e clientes. Para isto, implementaremos uma aplicação que irá realizar a comunicação por texto entre usuários, dos quais serão localizados para comunicação através de localização geográfica em um mapa terrestre real.

O produto mínimo viável da aplicação será elaborado para um ambiente de testes local, não oferecendo um fluxo completo de funcionamento dos servidores em ambiente de produção, nem do fluxo de utilização de um usuário comum em sua interface. Abordaremos apenas alguns conceitos sobre escalabilidade e infraestrutura da aplicação em ambiente de produção, assim como apenas algumas funcionalidades necessárias para interação na interface gráfica.

Iremos explicar com mais clareza os problemas que acercam este projeto no tópico "Problematização", onde falaremos especificamente quais são os ideais de implementação de uma solução em tempo real que tem características como as que nos propusemos a fazer, levando em conta aspectos como produto e serviço, já que ideologicamente um software é feito para realizar uma ou mais funções para um ou mais usuários, mas como citado, o nosso foco é demonstrar fundamentalmente o funcionamento deste tipo de aplicação, e não oferecer um produto ou serviço. A problematização irá se perfazer em torno do estudo de uma forma de se implementar uma solução tempo-real para a \textit{web} envolvendo duas funcionalidades que julgamos serem de boa representatividade para a circunstância.

	\chapter{Trabalhos Relacionados}
Com o objetivo de situar-se melhor em relação ao funcionamento da arquitetura e funcionalidades do sistema, analisamos alguns trabalhos que se assemelham em um ou mais aspectos da aplicação desenvolvida. Dentre estas, estão relacionadas propostas de implementação, que explicam o funcionamento da arquitetura de aplicações tempo-real, e também projetos que foram publicados como produto de mercado.

Iremos fazer uma breve análise de cada uma das plataformas, sendo que algumas análises serão fundamentadas em publicações que falam sobre o funcionamento destas, pois as mesmas não se encontram mais disponíveis para acesso. Resolvemos incluí-las em nossa documentação, pois são ferramentas que tentaram como produto final uma aplicação muito semelhante a que iremos desenvolvedor como caso de estudo.

\section{Uber}
Iniciando a pesquisa sobre o desenvolvimento de uma aplicação tempo-real, buscamos uma referência de grande escala \cite{uber-statistics}. O Uber \cite{uber} é uma aplicação \textit{web} e \textit{mobile} que oferece serviços de transporte urbano privado através de pessoas que se disponibilizam a serem motoristas e estarem disponíveis na plataforma para alguma oferta de demanda.

Encontramos uma publicação \cite{uber-how-scales} que fez-se de grande valia para o inicio da escolha das tecnologias que utilizamos para desenvolver a aplicação deste trabalho. A publicação fala sobre como era a arquitetura do sistema, para quais possibilidades ela foi pensada, e sobre a nova organização do sistema e quais tecnologias utilizam para suportar tamanha demanda \cite{uber-statistics}.

É notório ao longo do trabalho que a arquitetura aplicada para desenvolvimento da aplicação não se aproxima em muitos pontos da aplicada no Uber, mas após o estudo da publicação, fomos fomentados a pesquisar algumas das tecnologias utilizadas e como aplicá-las em nosso projeto, onde a principal influência foi a utilização do Redis como banco de dados em memória, e não diretamente uma influencia, mas uma consolidação de ideias, a utilização do NodeJS e um banco de dados NoSQL para persistência dos dados.

\section{Redis Publish-Subscribe (RedisMVA) }
Esta publicação \cite{redis-pubsub-redismva} faz parte de um projeto demonstrativo da utilização do padrão de comunicação pub/sub, com um banco de dados NoSQL, chamado Redis. A publicação fala sobre a implementação em NodeJS de um \textit{chat} de comunicação por texto, utilizando o padrão pub/sub para realizar a comunicação entre processos, onde os processos são as diversas instâncias que podem ser executadas de um mesmo servidor NodeJS distribuídos em diversos locais na internet.

O estudo desta publicação serviu como grande base de aprendizado para a elaboração da arquitetura do sistema que desenvolvemos, pois utilizava tecnologias atuais e que já tínhamos como plano para a implementação do nosso sistema, como citamos no caso do Uber. A utilização da linguagem de programação JavaScript, naturalmente traz um certo intimismo para quem já desenvolve para a \textit{web}, dado que ela pode ser considerada a linguagem da \textit{web} (\textit{client-side}), pois todos os maiores navegadores \textit{web} \cite{browsers-usage} utilizados no mundo interpretam a linguagem nativamente, sendo muito comum programadores voltados para a área já terem tido algum contato com a linguagem.

O autor da publicação disponibiliza todo o código fonte do projeto de demonstração com licença aberta, sendo assim, ele discorre sobre o tema de forma muito objetiva, apenas abordando em seu texto os pontos de interseção em que a implementação da comunicação pub/sub se insere. O texto oferece uma didática teórica de boa qualidade, demonstrando com diagramas e imagens da aplicação real, o funcionamento de cada passo do qual explica no documento.

A demonstração dos fluxos de dados que ocorrem no sistema ficam bem representadas com os diagramas, embora alguns destes fluxos não estejam implementados no sistema, como o balanceador de carga, visto que não era o foco de esclarecimento, é simples entender como a interface envia os dados para o servidor, que distribui os dados para os outros clientes conectados.

\section{Redis Pub-Sub (Rajaraodv)}
Esta publicação \cite{redis-pubsub-rajaraodv} fala sobre o mesmo assunto da citação anterior, que é a implementação de um chat por texto utilizando NodeJS e banco de dados Redis, mas nesta publicação o autor é mais abrangente no quesito de infraestrutura e escalabilidade da aplicação, abordando todos os passos para uma implantação completa do sistema em um servidor.

O texto da publicação é muito explicativo, é possível até para os mais principiantes no assunto de aplicações tempo-real entenderem a implementação e a finalidade de cada passo relatado. O texto também conta com ilustrações dos fluxos de dados na aplicação e demonstrações do funcionamento da interface.

O que se pode extrair do conteúdo desta publicação além da anterior, é a implementação do balanceador de carga, em conjunto com o \textit{proxy} reverso Nginx, o conceito de \textit{sticky sessions} para traçar requisições a um mesmo servidor, e a integração do sistema em um servidor na nuvem (conjunto de recursos computacionais na internet, dedicados a hospedar aplicações de diversos tipos). 

Após o estudo da publicação deste autor, utilizamos em nossa implementação a utilização do Nginx. Entendemos a ideia de centralizar a distribuição de conteúdo estático da aplicação, como a interface, e canalizar as conexões aos servidores através de um ponto em comum, facilitando assim o gerenciamento para futuras implementações de \textit{sticky sessions}, balanceamento de carga, reconexão em casos de \textit{scale-up} ou \textit{scale-out} dos servidores (aumento e diminuição do número de servidores), entre outros benefícios que este tipo de arquitetura pode prover para as aplicações modernas, aspirando o rumo em que a tecnologia vem tomando com o crescimento exponencial de usuários, consumo de dados, globalização, entre outros, que contribuem para a necessidade de se construir sistemas que sejam distribuídos e escaláveis, alcançando altos desempenhos.

\section{FanMappr e RadiusIM}
O FanMappr \cite{fanmappr} e o RadiousIM \cite{radiusim} são aplicações baseadas em \textit{chat} por localização geográfica, mas que foram disponibilizadas como produto, com a proposta de criarem relações sociais através da busca de usuários por localização, ou seja, são redes sociais e tinham isso como proposta final. Após uma análise mais técnica nos dois últimos tópicos em relação ao desenvolvimento da aplicação, agora iremos analisar ferramentas que implementam como um produto, aplicações que tem funcionalidades semelhantes as do nosso projeto.  

Encontramos o FanMappr e RadiusIM após algumas poucas pesquisas que buscavam especificamente produtos que tivessem funcionalidades muito próximas das que implementamos em nossa aplicação. O FanMappr é uma aplicação com interface e funcionalidades aparentemente modestas, navegando entre os poucos menus que possui, percebemos que as funcionalidades se restringem somente ao \textit{chat} por texto e a um mapa terrestre onde os usuários são localizados. O RadiusIM não está mais disponível para acesso, mas encontramos uma publicação \cite{radiusim} que possuía algumas imagens sobre a aplicação e uma breve explicação do seu funcionamento, que também é muito semelhante com do FanMappr, mas o RadiusIM, apesar de parecer ter tido mais sucesso, nos anos 2000, presume-se que teve um investimento maior, pois possuía uma interface mais harmoniosa e funcionalidades que eram o de se esperar para sua época.

É interessante perceber que estas aplicações não tiveram grande sucesso colocando como produto final uma ferramenta de comunicação com funcionalidades semelhantes a que iremos implementar, o que nos leva a interpretar que a \textit{web} em tempo real não é uma inovação a qualquer custo. É importante considerar que as melhorias que se pode ter com informações em tempo real devem ser mensuradas, e ela é apenas mais uma opção que está ganhando mais importância com as demandas que estão surgindo.

\section{Snapchat, Happn, Tinder e afins}
Neste tópico iremos fazer uma análise, mais filosófica do que técnica, sobre a utilização de aplicações tempo-real na vida dos usuários, pois achamos interessante fundamentar onde estão se rompendo os limites para a necessidade de aplicações deste tipo, e gostaríamos de ressaltar que Snapchat \cite{snapchat}, Happn \cite{happn} e Tinder \cite{tinder} foram escolhas que achamos que representam um grupo de aplicações com características próximas, mas que para muitas outras aplicações existentes, caberiam-se a mesma análise.

A seguinte citação \cite{notnotcitricsquid} nos traz algumas importantes informações sobre a opinião de um usuário comum:

\begin{quote}
	\small "There's a fundamental difference between location based chat and chat applications like Snapchat. Snapchat changed how people communicate, it did not change how they found people to communicate with... it may have helped better relationships form but it did not create them out of nothing.
	
	Arbitrary things like age and location don't define who we are or what we enjoy, and who we are and what we enjoy is a big part of whether or not we will enjoy being friends with someone. Relationships are an investment, most people don't want to talk to a bunch of new people every day but never form any lasting relationship. If they want to form a lasting relationship it needs to be based on more than "I saw you at whole foods"... maybe their phone will ping in whole foods and maybe it'll be someone interesting and maybe they'll become friends and maybe it'll be a fantastic relationship, but what are the chances of that happening? How many people are willing to go through hundreds and hundreds of worthless chats just to find one friend?
	
	Maybe one day it will be time for a location based chat app to take off, but I don't think that time is now." 
	
	("Do location based chat apps start trending again?" \cite{notnotcitricsquid}, NOTNOTCITRICSQUID, 2012)
\end{quote}

A resposta a pergunta do tópico expressa uma opinião que condiz com alguns pontos que pudemos notar ao longo do conteúdo apresentado até aqui, que se relacionam com como as aplicações tem obtido sucesso na inserção da experiência de tempo real na vida dos usuários mais comuns, e aplicações como o FanMappr e o RadiusIM não alcançaram o mesmo resultado. Snapchat, Happn e Tinder são aplicações que funcionam baseando parcialmente e/ou totalmente suas funcionalidades em \textit{chat} por localização geográfica, mas diferem-se com uma certa cautela da experiência exposta aos seus usuários nas suas aplicações, e este é um ponto que acreditamos ser o diferencial para a aceitação das ferramentas, pelo fato não infringir algumas barreiras que existem na interação humana com o mundo virtual.

A possibilidade da informação instantânea tem muito a contribuir em diversas áreas, oferecendo vantagem competitiva para quem está no mundo dos negócios, antecipação de tomada de decisões, entre outros benefícios. Este mundo do tempo-real está a cada vez mais tomando conta do nosso presente como usuários comuns, e a tendência é que essa forma de compartilhamento instantâneo se torne mais natural.

Como as pessoas utilizam e disponibilizam dados, tem uma importância fundamental no impacto em que isso pode ter em suas vidas. Desde as necessidades no mundo dos negócios, às necessidades do cotidiano de cada um. Por parte de quem está inovando no setor, é preciso entender comportamentos e como não romper barreiras que irão fracassar a relação entre tecnologia e pessoas. 

Fazemos a ressalva de que a nossa implementação não busca oferecer funcionalidades que atendam aos requisitos que entendemos como necessários em relação a estes conceitos, mas abordamos o assunto pelo interesse em mostrar como a tecnologia também está envolvida com as relações humanas.

	\chapter{Problematização}
\section{Caracterização do Problema}
A internet oferece diversas possibilidades de integração e distribuição de dados em sua rede mundial. A necessidade cada vez maior de se coletar dados que são gerados a todo momento, cria a iniciativa de se laborar novas maneiras de envio e recepção de dados na rede.

A transmissão de dados em tempo real não é uma necessidade recente, desde que se começaram a desenvolver sistemas críticos de controle na \textit{web}, muitos destes já utilizavam desta forma de se transmitir dados para realizar suas tarefas. As técnicas e tecnologias utilizadas para alcançar tal feito eram diferentes e mais custosas, o que levava este tipo de característica da aplicação ser adotada apenas em casos específicos, onde deveria se contrabalancear com mais perícia a utilização.

Com os avanços das tecnologias de \textit{hardware}, \textit{software} e da internet como um todo, começaram a se criar novas possibilidades da implementação de \textit{features} tempo-real em aplicações cada vez mais próximas do usuário comum. Com tudo isso, podemos observar que muitas das aplicações de sucesso do mercado usufruem desta tecnologia, que permite uma conexão direta e persistente entre cliente e servidor, viabilizando que se possa receber e enviar dados a qualquer momento, atualizando em tempo real todos os interessados em uma determinada informação.

A possibilidade da informação instantânea tem muito a contribuir em diversas áreas, oferecendo vantagem competitiva para quem está no mundo dos negócios, antecipação de tomada de decisões, entre outros benefícios. Este mundo do tempo-real está a cada vez mais tomando conta do nosso presente como usuários comuns, e a tendência é que essa forma de compartilhamento instantâneo se torne mais natural.

Como as pessoas utilizam e disponibilizam dados, tem uma importância fundamental no impacto em que isso pode ter em suas vidas. Desde as necessidades no mundo dos negócios, às necessidades do cotidiano de cada um. Por parte de quem está inovando no setor, é preciso entender comportamentos e como não romper barreiras que irão fracassar a relação entre tecnologia e pessoas. 

A implementação de funcionalidades tempo-real definitivamente se tornou algo exequível para as aplicações atuais na \textit{web}, como \textit{push notifications} (notificações de eventos que são emitidas do servidor para o cliente), \textit{stats} (status de dados), interações em jogos de navegadores, entre outros. Os \textit{WebSockets} funcionam analogicamente como os \textit{sockets}, e nos permite fazer a comunicação entre processos através da rede. Como o próprio nome diz, os \textit{WebSockets} são especificamente feitos para este tipo de comunicação na \textit{web}, possibilitando uma conexão contínua entre códigos JavaScript executados nos \textit{browsers} e os servidores \text{web}. Estes serão base fundamental para a resolução do problema proposto neste trabalho, levando em conta que são a tecnologia mais atual.

\section{Solução Proposta}
Buscando explorar a usabilidade tempo-real na \textit{web}, iremos formular nossa solução no desenvolvimento de uma aplicação de comunicação por texto, também podendo ser chamada de \textit{chat}. Para que os usuários tenham conhecimento da existência um do outro na aplicação, existirá um mapa terrestre do mundo real que será marcado com pontos identificadores de cada usuário utilizando o sistema. Estes pontos terão como base a coordenada geográfica real do usuário que será fornecida quando o usuário entrar no sistema.

A escolha por uma aplicação deste modelo se deve a ideia de querer tornar o entendimento do emprego da tecnologia mais amigável para as pessoas que não possuem conhecimento técnico muito aprofundado. A troca mensagens explícita em uma conversa entre duas pessoas, essencialmente ilustra a ideia de transferência de dados em tempo real, e em conjunto com a localização geográfica, conseguimos dar um adendo de outra aplicação da tecnologia.

Iremos desenvolver uma interface com a pretensão de ser amigável ao usuário, utilizando conceitos de responsividade para a adaptação do conteúdo mostrado em telas de dispositivos diversos, inclusive móveis, tendo em vista que algumas estatísticas \cite{internet-traffic-stats1} apontam que dispositivos móveis já são responsáveis por pelo menos um quarto do todo o tráfego de dados gerado na internet, e com estimativas \cite{internet-traffic-stats2} \cite{internet-traffic-stats3} de aumento em cada ano. Mesmo não sendo o foco da aplicação ser um produto final, achamos que o aprendizado com este tipo de requisito seria conveniente. Em relação a organização da interface, a aplicação terá duas telas principais, sendo a primeira com as opções de cadastro através do fornecimento de alguns dados pessoais, e autenticação no sistema por \textit{e-mail} e senha. A segunda tela será o ambiente principal da aplicação, onde o usuário poderá navegar por um mapa terrestre em busca de marcações que identifiquem outro usuário, com o qual ele poderá interagir através de uma janela de \textit{chat} por texto que surge neste mesmo ambiente.

O principal componente da solução será o servidor da aplicação. Este será desenvolvido sobre a arquitetura REST (Representational State Transfer), que transmite e recebe dados da interface no formato JSON (JavaScript Object Notation). Funcionalidades como cadastro de usuário e autenticação farão uso deste modelo. A comunicação tempo-real será provida por um \textit{framework} que utiliza como base a tecnologia de WebSockets para conexões bidirecional de dados entre servidor e cliente. Apesar da tecnologia WebSocket ser a mais apropriada para a este tipo de conexão, o próprio \textit{framework} trata de usar outras técnicas de comunicação tempo-real, como \textit{long polling}, em casos de falta de compatibilidade dos navegadores com a tecnologia de WebSocket. Tudo isso acontece de forma transparente ao usuário e ao desenvolvedor, pois a utilização do \textit{framework} se dá através de API (Application Programming Interface) que oferece de forma simplificada todo o processo de conexão e comunicação.

A persistência dos dados será feita através de um banco de dados no NoSQL. A escolha por este tipo de banco de dados se deve as características da aplicação que exigem muitas operações de I/O (Input/Output) em dados com baixo nível de relacionamento com outros dados, onde este tipo de tecnologia NoSQL oferece melhor desempenho e praticidade.

A tecnologia utilizada como ambiente de execução no servidor, utiliza o JavaScript como linguagem padrão, além de oferecer características que também beneficiam o tipo de aplicação que iremos desenvolver, como \textit{single-theaded} (aplicação executada em apenas uma thread, processando um comando por vez), \textit{non-blocking IO} (operações de escrita e leitura de dados não bloqueiam a execução de código) e \textit{asynchronism} (não existe ordem de execução dos blocos de código). 

Considerando também alguns conceitos de infraestrutura, iremos adotar na arquitetura da aplicação algumas formas de permitir o escalonamento dos serviços através da distribuição dos servidores em diversas instâncias no mesmo servidor físico ou em outros de locais distintos. Para isso utilizaremos tecnologias para fazer a comunicação entre processos, com um banco de dados NoSQL centralizado e em memória, e interações \textit{stateless}, onde o servidor não guarda dados de sessão do usuário.
	\chapter{Hardware e Software Utilizados}

\section{Codificação e Testes}
\subsection{Ambiente de Desenvolvimento}
O ambiente para desenvolvimento da solução foi criado em um único computador, de caráter pessoal, do tipo notebook, com as seguintes especificações técnicas \cite{notebook-info}:
	\begin{itemize}
		\item Processador Intel Core i5 3230M - 2.60 GHz (T-Max: 3.20 GHz) - 64 Bits - Smart Cache 3 MB L3
	\end{itemize}
	\begin{itemize}
		\item Chipset Intel ® HM75 Express
	\end{itemize}
	\begin{itemize}
		\item Memória RAM 8 GBs DDR3 SDRAM - 1600 MHz
	\end{itemize}
	\begin{itemize}
		\item Disco rígido 500GB - 5.400 RPM
	\end{itemize}
	\begin{itemize}
		\item Rede 10/100 e Wireless 802.11 b/g/n - Link de conexão com a internet de 15 Mbps
	\end{itemize}

Especificamos apenas algumas informações que julgamos serem de relevância para alguma análise de desempenho da aplicação por parte do leitor. Em nossa solução não iremos realizar levantamentos de dados sobre desempenho, pois este não pode ser considerado um ambiente que simule a realidade em servidores de aplicações, do qual seria de relevância este tipo de medição.

O sistema operacional escolhido para nosso ambiente de desenvolvimento foi o CentOS 7. O CentOS é uma distribuição Linux baseada no Red Hat. A Red Hat é uma empresa privada, que possui um O.S. chamado Red Hat Enterprise Linux (RHEL), e tem como proposta oferecer produtos \textit{enterprise} baseados no GNU/Linux. Tendo em conta que toda distribuição baseada no \textit{core} do Linux deve seguir uma série de regras contidas em sua licença, a Red Hat concentra seus faturamentos em suporte as distribuições que ela construiu. Ter um sistema compatível com o RHEL traz alguns benefícios que ainda não foram alcançados sem que se tenha um time de pessoas dedicadas ao desenvolvimento de um \textit{software} específico, como no caso da maioria dos sistemas open-sources, que é o alto padrão em testes contra erros e problemas de segurança, \textit{releases} mais curtos, correções de problemas em curto prazo, entre outros benefícios que o CentOS consegue absorver da comunidade Red Hat. Além disso, o CentOS já possui uma considerável comunidade ativa, que facilita na resoluções de problemas mais específicos e no desenvolvimento de novas funcionalidades. O sistema também já possui grandes \textit{cases} de sucesso, como Facebook \cite{facebook-distro}, Google \cite{google-redhat}, entre outros, o que passa uma maior confiança para a adoção em outros negócios e uma perspectiva de crescimento contínua.

\subsection{Editores de Texto}
Para o desenvolvimento majoritário do \textit{frontend}, utilizamos o editor de texto SublimeText 3 \cite{sublime}, apoiado de alguns \textit{plugins} para auxílio de escrita e design, como Emmit, ColorHighlighter, AlignTab, JSHint, MaterialTheme entre outros, todos podendo ser encontrado no repositório de pacotes packagecontrol.io \cite{packagecontrolio}.

Para o desenvolvimento do \textit{backend} e parte do \textit{frontend}, utilizamos o editor de texto Visual Studio Code 1.6 \cite{vscode}. A escolha por este outro editor de texto se fez principalmente pela funcionalidade de \textit{debug}, que apresentou melhores resultados em sua utilização, além de ser uma funcionalidade nativa da aplicação, assim como a ferramenta para versionamento de código. Também foi utilizado \textit{plugins} semelhantes aos escolhidos no SublimeText para auxílio de escrita de código, todos encontrados no repositório de pacotes padrão do editor.

\section{Back-end}
\subsection{NodeJS}


\subsection{NPM}


\subsection{ExpressJS}


\subsection{SocketIO}


\subsection{MongoDB}


\subsection{Redis}


\section{Front-end}

\subsection{Bower}


\subsection{AngularJS}


\subsection{Angular Material}


\subsection{Gulp}


\subsection{SASS}


\subsection{Webpack}


\subsection{Babel}


\subsection{Nginx}
	\chapter{Implementação}
A implementação do sistema segue todas as diretrizes de tecnologias e conceitos apresentados até este tópico. A descrição de como a implementação foi feita será dividida em duas partes principais: \textit{backend} e \textit{frontend}. Sendo a parte de \textit{backend} a implementação da infraestrutura e servidor da aplicação, e na parte do \textit{frontend} a implementação de interface com o usuário. No desenvolvimento elegemos um nome fictício para a aplicação, chamado Orb. Este nome será utilizado para remeter a aplicação em alguns textos, logos, implementações, entre outros.

O código-fonte da aplicação está hospedado no GitHub \cite{orb}, como \textit{software open-source}, liberado para utilização através da licença MIT \cite{mit}. A abordagem utilizada para a descrição da implementação feita fará referências pontuais a trechos do código-fonte que poderão ser consultados pelo leitor no repositório informado \cite{orb}, evitando-se ao máximo a exposição de código neste formato de texto que não é favorável a análise deste tipo de conteúdo.

\section{Backend}
Primeiramente iremos descrever a criação do ambiente de desenvolvimento do projeto, onde inicialmente foi instalado o sistema operacional e todas as ferramentas que decidimos utilizar para a construção da aplicação, e posteriormente iniciamos o processo de codificação.

Iniciando-se pelo sistema operacional, foi criada uma partição separada no disco rígido do computar que dispúnhamos, com volume de 60 GBs. O sistema operacinal escolhido para o ambiente de desenvolvimento foi de uma distribuição Linux chamada CentOS \cite{centos}, versão 7. A instalação do sistema operacional foi feita através de um \textit{pendrive} e foram seguidas todas as configurações padrões do \textit{wizard} de instalação do sistema.

Após a instalação do sistema operacional, fizemos a instalação de todas as ferramentas, a nível de O.S., que foram escolhidas para o desenvolvimento da aplicação, como o NodeJS, MongoDB, Redis, editores de texto e navegadores. Todas estas ferramentas foram instaladas através do repositório de pacotes padrão do sistema CentOS 7, chamado YUM. Para algumas ferramentas foram necessário a atualização das fontes do repositório YUM e outras foram instaladas através de pacotes RPM disponibilizados no próprio site da ferramenta.

A implementação do \textit{backend} iniciou-se com a criação de um diretório chamado backend na pasta raiz do projeto, do qual a partir dele foram seguidas recomendações de boas práticas \cite{express-app-structure} para a organização da estrutura de diretórios em aplicações ExpressJS. A seguinte listagem identifica o nome destes diretórios e suas respectivas funcionalidades:

\begin{description}
	\item[controllers] Este diretório contém os arquivos de código relacionados as rotas e suas lógicas de implementação. Rotas são os \textit{endpoints} que podem ser acessados através de requisições HTTP enviadas por uma determinada URL que a identifica. Neste diretório contém os arquivos das rotas de cliente, usuário e \textit{token}.
	
	\item[domain] Este diretório contém os arquivos relacionados a lógica de negócio da aplicação. Os \textit{handlers} que tratam as requisições \textit{websocket} da aplicação estão implementados em arquivos contidos nesta pasta.
	
	\item[helpers] Este diretório contém arquivos que implementam funcionalidades que podem ser utilizadas em diversos pontos do sistema, este diretório funciona como um \textit{cross-cutting layer} da aplicação. Nesta pasta estão implementados alguns métodos que ajudam na geração de \textit{random strings} e retorno de \textit{datetime} em formato UTC.
	
	\item[middlewares] Este diretório contém arquivos relacionados a \textit{middlewares} de tratamento de requisições que chegam ao servidor, como realização de autenticação e autorização de acesso as rotas da aplicação.
	
	\item[models] Este diretório contém arquivos relacionados aos objetos que podem ser persistidos na aplicação. Devido ao Javascript ser uma linguagem dinâmica, estes objetos não representam um contrato de como estes eles devem existir na aplicação, mas como eles devem ser persistidos no banco de dados, que no caso desta aplicação é o MongoDB. Neste diretório serão encontrados os arquivos de persistência dos \textit{tokens}, \textit{chat}, \textit{messages}, \textit{users}, entre outros.
\end{description}

Após a definição da estrutura de diretórios da aplicação, foi realizada a instalação e inicialização do NPM, responsável por gerenciar todos os pacotes que iremos utilizar no desenvolvimento da solução.

\subsection{Persistência}
A persistência dos dados foi feita em um banco de dados MongoDB, onde para sua utilização foi feito o \textit{download} doo pacote Mongoose pelo NPM. A conexão com o banco foi feita no arquivo de inicialização da aplicação, através do método connect do próprio Mongoose, passando-se como parâmetros uma \textit{connection string} com o endereço local do banco de dados. Por ser um ambiente de testes, não utilizamos autenticação com o banco de dados, mas caso fosse necessário, bastava-se apenas adicionar os dados de \textit{username:password} na própria \textit{connection string}.

No diretório \textit{models} foram criados arquivos para cada objeto que fosse ser persistido no banco. As implementações da persistência utilizaram o Mongoose como modelo, exportando de cada módulo o objeto \textit{schema} de cada persistência, podendo através deste realizar todas as operações de CRUD.

A seguinte imagem ilustra a representação dos objetos persistidos no banco de dados:

\begin{figure}[!htb]
	\centering
	\includegraphics[scale=0.85]{imagens/models_uml.png}
	\caption{\small Representação UML dos modelos persistidos no banco de dados.}
	\label{fig:modelsuml}
\end{figure}


\subsection{Autenticação e Autorização}
Para a implementação do sistema de autenticação, cujos códigos estão no diretório middlewares, utilizamos três pacotes NPM, que funcionam como \textit{middlewares} que extraem de um \textit{request} as informações de autenticação, e através de uma \textit{callback} é possível validar essas informações e sinalizar se as informações estão corretas para concretização da autenticação. Para o sistema de autorização, utilizamos o OAuth2 com algumas modificações em seu padrão para atender aplicações \textit{trusted}, como o nosso \textit{frontend app} desenvolvido em AngularJS, que é um tipo de aplicação que não necessita de autorização do usuário da conta para acessar suas informações confidenciais.

Como estamos utilizando um sistema de \textit{tokens}, para realizar acesso ao \textit{endpoint} token, o qual é utilizado no sistema de \textit{login} para se conseguir um Access Token e um Refresh Token através do fornecimento de \textit{e-mail} e senha, é realizada a autenticação pelos padrões Basic \cite{passport-basic} e Client \cite{passport-client}. Após esta autenticação, o usuário é encaminhado para um \textit{exchange} do OAuth2 que irá gerar, persistir e retornar um Access Token e um Refresh Token para o solicitante, que poderá usá-lo para acessar os demais \textit{endpoints} da aplicação.

Como já introduzido, o restante dos das rotas da aplicação são protegidos através do padrão de autenticação Bearer \cite{passport-bearer}, que procura em cada requisição feita um Access Token, que será verificado, e caso seja válido permitirá o acesso ao \textit{endpoint} solicitado. Em caso do Access Token não ser válido, o servidor irá emitir uma resposta de erro, HTTP 401, informando o problema ocorrido, e se o problema for a expiração do Access Token, ele poderá utilizar o Refresh Token para conseguir um novo Access Token válido por mais um determinado tempo, evitando de ter que realizar toda o processo de autenticação com \textit{e-mail} e senha novamente.

\begin{figure}[!htb]
	\centering
	\includegraphics[scale=0.49]{imagens/oauth2.png}
	\caption{\small Fluxo de acesso aos recursos com OAuth2. Fonte: jlabusch \cite{img-jlabusch}}
	\label{fig:oauth2}
\end{figure}

\subsection{Rotas}


\subsection{Socket}


\section{Frontend}


	\chapter{Resultados}
O principal objetivo do trabalho foi alcançado na implementação do sistema, que foi explorar o funcionamento de aplicações de tempo real. Com a ajuda de todas as ferramentas citadas, foi possível desenvolver uma aplicação \textit{web} com alguns fluxos de utilização que facilitam o entendimento do que é e como pode ser aplicada esse tipo de tecnologia na construção de uma ferramenta de comunicação, além do estudo de uma arquitetura que pode ser utilizada para outros tipos de finalidade. 

Para o projeto entregue foram alcançados a comunicação por mensagens de texto entre dois usuários, a exibição e alteração da localização real do usuário, o sistema de autenticação \textit{stateless} através de \textit{tokens}, criação de novas contas de usuário, persistência de mensagens no banco de dados e utilização de conceitos de escalabilidade em aplicações. 

As arquitetura do sistema seguiu conceitos de escalabilidade os quais não puderam ser testados em sua totalidade devidas a não extensão do tempo de elaboração do trabalho final, apenas demonstramos teoricamente a função de cada tecnologia em um ambiente que se possa ter diversas instâncias da aplicação distribuídas em uma rede de servidores. Assim como a proficiência em desempenho também não foi alvo de análise.

Alguns componentes de interface, apesar de existirem visualmente, não foram implementadas as funcionalidades para a execução das tarefas, como as alterações de informações da conta do usuário, lista de contatos e página de sobre. Apesar da utilização de um \textit{framework} que provê compatibilidade e responsividade de componentes entre diversos \textit{browsers}, alguns componentes foram desenvolvidos a parte, e não foram realizados testes \textit{cross-browsers} para verificar o funcionamento correto de todos os componentes, sendo as funcionalidades apenas validadas no navegador de testes para desenvolvimento, o Google Chrome, versão 54.
	\chapter{Trabalhos Futuros}
A criação das funcionalidades de alterações de informações da conta do usuário e lista de contatos. assim como a validação dos componentes da interface em outros navegadores, como o FireFox, Opera, Safari, Internet Explorer e Microsoft Edge, podem ser implementações rápidas, devido a toda a infraestrutura já estar pronta, assim como a interface, e tornar o sistema útil para algum fim específico que possa se beneficiar das características de comunicação do sistema.

Pensando na utilização por diversos usuários do sistema desenvolvido, existem implementações que devem ser refeitas devido a problemas aparentes de desempenho que a aplicação pode ter, principalmente nas implementações de interação com o mapa, pois a aplicação atual marca todos os usuários \textit{online} no sistema no mapa de cada usuário, o que não é o ideal. Melhorias nesse sentido podem envolver uma pesquisa mais aprofundada nas APIs do Google Maps, para entender melhor as formas de utilização, assim como citamos em trabalhos relacionados, o qual o Uber pode utilizar ferramentas diferentes diferentes para a elaboração do seu mapa. 

Outro ponto importante que envolve este problema de escalabilidade com o sistema de localização, pode ser a utilização das funcionalidades de geolocalização do banco de dados Redis, onde as informações ao invés de serem distribuídas uniformemente para todos os usuários, elas podem ser baseadas na localização do usuário requerente, onde por exemplo, os usuários obtenham informações apenas de outros usuários próximos a ele, fazendo-se cálculos para retorno de dados baseando-se em um determinado raio de distância.

A simulação de ambientes distribuídos e testes de estresse podem ser realizados na aplicação e trazerem resultados interessantes, que podem demonstrar a eficiência da arquitetura desenvolvida para a aplicação. Isto também implicará em novas implementações, como \textit{sticky sessions} e \textit{load balancer}, desmistificando muitos conceitos envolvidos em aplicações que suportam milhares de usuários simultâneos.
	\chapter{Conclusão}
A elaboração deste projeto trabalhou diversas áreas do desenvolvimento de \textit{softwares}, desde a infraestrutura à criação de interfaces. A aplicação de uma tecnologia que proporciona interação em tempo real, foco do nosso tema, foi facilmente introduzida devido as características que as tecnologias escolhidas já propunham, tendo em vista que a \textit{stack} MEAN (MongoDB, ExpressJS, AngularJS, NodeJS) é muito utilizada para a construção de aplicações de tempo real, possibilitando uma pesquisa muito mais fácil devido as diversas soluções semelhantes já elaboradas por outros desenvolvedores.

A exploração deste tema nos proporcionou conhecer melhor muitas das aplicações deste tipo de tecnologia, e principalmente a experiência em se desenvolver uma aplicação deste tipo, tendo em vista a solução que este tipo de tecnologia pode oferecer para diversos problemas tecnológicos. Os trabalhos relacionados apresentam grandes ferramentas que utilizam uma ou mais tecnologias citadas e demonstram a relevância do tema para o tempo atual da tecnologia, levando a trabalhos cada vez maiores e que exploram as diversas complexidades que sistemas de tempo real podem alcançar para atender as necessidades demandadas. Acreditamos que os resultados alcançados atingiram as metas propostas pelo trabalho, e deixam abertas oportunidades para pesquisas mais específicas da tecnologia.

A ideia deste trabalho foi ser como um passo inicial para o entendimento do ambiente de tempo real em aplicações, consequentemente não aplicamos e nem nos aprofundamos em algumas boas práticas, como o desenvolvimento orientado a testes, o que tornou a utilidade da aplicação meramente demonstrativa, mas que com mais trabalho pode se tornar algo utilizável para outros futuros estudos na área em que abrange.

\section{Trabalhos Futuros}
A criação das funcionalidades de alterações de informações da conta do usuário e lista de contatos. assim como a validação dos componentes da interface em outros navegadores, como o FireFox, Opera, Safari, Internet Explorer e Microsoft Edge, podem ser implementações rápidas, devido a toda a infraestrutura já estar pronta, assim como a interface, e tornar o sistema útil para algum fim específico que possa se beneficiar das características de comunicação do sistema.

Pensando na utilização por diversos usuários do sistema desenvolvido, existem implementações que devem ser refeitas devido a problemas aparentes de desempenho que a aplicação pode ter, principalmente nas implementações de interação com o mapa, pois a aplicação atual marca todos os usuários \textit{online} no sistema no mapa de cada usuário, o que não é o ideal. Melhorias nesse sentido podem envolver uma pesquisa mais aprofundada nas APIs do Google Maps, para entender melhor as formas de utilização, assim como citamos em trabalhos relacionados, o qual o Uber pode utilizar ferramentas diferentes diferentes para a elaboração do seu mapa. 

Outro ponto importante que envolve este problema de escalabilidade com o sistema de localização, pode ser a utilização das funcionalidades de geolocalização do banco de dados Redis, onde as informações ao invés de serem distribuídas uniformemente para todos os usuários, elas podem ser baseadas na localização do usuário requerente, onde por exemplo, os usuários obtenham informações apenas de outros usuários próximos a ele, fazendo-se cálculos para retorno de dados baseando-se em um determinado raio de distância.

A simulação de ambientes distribuídos e testes de estresse podem ser realizados na aplicação e trazerem resultados interessantes, que podem demonstrar a eficiência da arquitetura desenvolvida para a aplicação. Isto também implicará em novas implementações, como \textit{sticky sessions} e \textit{load balancer}, desmistificando muitos conceitos envolvidos em aplicações que suportam milhares de usuários simultâneos.

	
	%===================================================================================
	%\backmatter
	%===================================================================================
	
	%\bibliography{monografia}{}
	%\bibliographystyle{abnt-alf}
	
	% OUTRA FORMA DE CRIAR A BIBLIOGRAFIA:
	\begin{thebibliography}{9}
		\bibitem{startup-email-innovation1}Liveclicker Delivers Dynamic Messaging Capability with RealTime Email Solution. Disponível em: \url{http://www.destinationcrm.com/Articles/CRM-News/CRM-Featured-News/Liveclicker-Delivers-Dynamic-Messaging-Capability-with-RealTime-Email-Solution-97439.aspx}.
		Acesso em: 19 Out. 2016.
		
		\bibitem{startup-email-innovation2}Liveclicker RealTime Email Solution. Disponível em: \url{http://www.realtime.email/}.
		Acesso em: 19 Out. 2016.
		
		\bibitem{uber-statistics}50 Amazing Uber Statistics (October 2016). Disponível em: \url{http://expandedramblings.com/index.php/uber-statistics/}.
		Acesso em: 20 Out. 2016.
		
		\bibitem{uber}Uber. Disponível em: \url{https://www.uber.com/}.
		Acesso em: 20 Out. 2016.
		
		\bibitem{uber-how-scales}How Uber Scales Their Real-Time Market Platform. Disponível em: \url{http://highscalability.com/blog/2015/9/14/how-uber-scales-their-real-time-market-platform.html}.
		Acesso em: 20 Out. 2016.
		
		\bibitem{redis-pubsub-redismva}Redis Publish-Subscribe (RedisMVA). Disponível em: \url{https://github.com/sayar/RedisMVA/blob/master/module6_redis_pubsub/README.md#redis-publish-subscribe}.
		Acesso em: 20 Out. 2016.
		
		\bibitem{browsers-usage}Web browsers usage table. Disponível em: \url{http://caniuse.com/usage-table}.
		Acesso em: 20 Out. 2016.
		
		\bibitem{redis-pubsub-rajaraodv}Redis Pub/Sub (Rajaraodv). Disponível em: \url{https://github.com/rajaraodv/redispubsub}.
		Acesso em: 20 Out. 2016.
		
		\bibitem{fanmappr}FanMappr. Disponível em: \url{http://h.fanapp.mobi/modules/fanmappr/fanmappr.php?fid=1056666}.
		Acesso em: 20 Out. 2016.
		
		\bibitem{radiusim}RadiusIM – Social Messenger Com Busca Por Geotagging. Disponível em: \url{http://br.wwwhatsnew.com/2008/09/radiusim-%E2%80%93-social-messenger-por-com-busca-por-geotagging/}.
			Acesso em: 20 Out. 2016.
			
		\bibitem{notnotcitricsquid}NOTNOTCITRICSQUID. Do location based chat apps start trending again?. Disponível em: \url{https://www.reddit.com/r/startups/comments/1tkc3f/do_location_based_chat_apps_start_trending_again/}.
		Acesso em: 21 Out. 2016.
		
		\bibitem{snapchat}Snapchat. Disponível em: \url{https://www.snapchat.com/}.
		Acesso em: 21 Out. 2016.
		
		\bibitem{tinder}Tinder. Disponível em: \url{https://www.gotinder.com/}.
		Acesso em: 21 Out. 2016.
		
		\bibitem{happn}Happn. Disponível em: \url{https://www.happn.com/}.
		Acesso em: 21 Out. 2016.
		
		\bibitem{internet-traffic-stats1}How Much Web Traffic Is On Mobile Devices (Really)? Disponível em: \url{https://keriganmarketing.com/freelunch/view/how-much-web-traffic-is-on-mobile-devices/}.
		Acesso em: 22 Out. 2016.
		
		\bibitem{internet-traffic-stats2}Internet stats and facts for 2016. Disponível em: \url{https://hostingfacts.com/internet-facts-stats-2016/}.
		Acesso em: 22 Out. 2016.
		
		\bibitem{internet-traffic-stats3}Mobile Marketing Statistics compilation. Disponível em: \url{http://www.smartinsights.com/mobile-marketing/mobile-marketing-analytics/mobile-marketing-statistics/}.
		Acesso em: 22 Out. 2016.
		
		\bibitem{notebook-info}Notebook HP 1000-1460BR. Disponível em: \url{http://www.lojahp.com.br/Voce/Notebook/Notebook-HP-1000-1460BR-com-Intel-Core-i5-3230M-Windows-8-4GB-500GB-Gravador-de-DVD-Leitor-de-Cartoes-HDMI-e-LED-14-2177574.html}.
		Acesso em: 24 Out. 2016.
		
		\bibitem{facebook-distro}Facebook Linux, What Distro is it?. Disponível em: \url{http://www.internetnews.com/blog/skerner/facebook-linux-CentOS.html}.
		Acesso em: 24 Out. 2016.
		
		\bibitem{google-redhat}Red Hat Enterprise Linux Atomic Host expands to Google Compute Engine. Disponível em: \url{https://cloudplatform.googleblog.com/2014/07/red-hat-enterprise-linux-atomic-host-expands-to-google-compute-engine.html?m=1}.
		Acesso em: 24 Out. 2016.
		
		\bibitem{sublime}Sublime Text. Disponível em: \url{https://www.sublimetext.com/}.
		Acesso em: 24 Out. 2016.
		
		\bibitem{packagecontrolio}Package Control. Disponível em: \url{https://packagecontrol.io/}.
		Acesso em: 24 Out. 2016.
		
		\bibitem{vscode}Visual Studio Code. Disponível em: \url{https://code.visualstudio.com/}.
		Acesso em: 24 Out. 2016.
		
	\end{thebibliography}
		\anexo
	\end{document}