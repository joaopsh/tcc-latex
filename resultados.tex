\chapter{Resultados}
O principal objetivo do trabalho foi alcançado na implementação do sistema, que foi explorar o funcionamento de aplicações tempo-real. Com a ajuda de todas as ferramentas citadas, foi possível desenvolver uma aplicação \textit{web} com alguns fluxos de utilização que facilitam o entendimento do que é e como pode ser aplicada esse tipo de tecnologia na construção de uma ferramenta de comunicação, além do estudo de uma arquitetura que pode ser utilizada para outros tipos de finalidade. 

Para o projeto entregue foram alcançados a comunicação por mensagens de texto entre dois usuários, a exibição e alteração da localização real do usuário, o sistema de autenticação \textit{stateless} através de \textit{tokens}, criação de novas contas de usuário, persistência de mensagens no banco de dados e utilização de conceitos de escalabilidade em aplicações. 

As arquitetura do sistema seguiu conceitos de escalabilidade os quais não puderam ser testados em sua totalidade devidas a não extensão do tempo de elaboração do trabalho final, apenas demonstramos teoricamente a função de cada tecnologia em um ambiente que se possa ter diversas instâncias da aplicação distribuídas em uma rede de servidores. Assim como a proficiência em desempenho também não foi alvo de análise.

Alguns componentes de interface, apesar de existirem visualmente, não foram implementadas as funcionalidades para a execução das tarefas, como as alterações de informações da conta do usuário, lista de contatos e página de sobre. Apesar da utilização de um \textit{framework} que provê compatibilidade e responsividade de componentes entre diversos \textit{browsers}, alguns componentes foram desenvolvidos a parte, e não foram realizados testes \textit{cross-browsers} para verificar o funcionamento correto de todos os componentes, sendo as funcionalidades apenas validadas no navegador de testes para desenvolvimento, o Google Chrome, versão 54.