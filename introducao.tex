% Capítulo
\chapter{Introdução}
Os avanços da tecnologia em conjunto com a internet têm ditado aos seus usuários novas formas de consumo de dados e informações. Além disto, um assunto tem ganhado cada vez mais destaque no provimento de dados na rede: a periodicidade. A inclusão dos meios tecnológicos no cotidiano das pessoas e o grande aumento do volume de dados gerados têm desenvolvido uma nova tendência de informações instantâneas, o que nos leva a elaborar meios para atender a esta nova demanda.

Alguns questionamentos aos modelos atuais de consumo de dados,  apesar de bem sucedidos em diversas áreas, surgem como uma necessidade ainda maior pelo consumo de informações. O \textit{e-mail}, tradicional meio de comunicação entre pessoas, oferecendo uma boa razão de confiabilidade e formalidade, já é questionado por \textit{startups} \cite{startup-email-innovation1} \cite{startup-email-innovation2} que acreditam no futuro da \textit{web} em tempo real.

% Seção
\section{Motivação} 
As aplicações de tempo real estão surgindo como requisitos de muitos dos novos sistemas em desenvolvimento, e buscam atender consumidores que já possuem em seus negócios a necessidade da informação imediata. Neste trabalho será desenvolvido uma aplicação que explora de forma bem específica essa necessidade da troca de informações entre usuário e aplicação em tempo real.

A aplicação desenvolvida será um \textit{web chat} de comunicação por texto baseado em localização geográfica. As duas principais características do projeto, comunicação por texto e localização geográfica, trazem para si aspectos modernos, que exigem a utilização de ferramentas e técnicas atuais de desenvolvimento para a \textit{web}. Deve-se levar em consideração diversos conceitos, como compatibilidade, escalabilidade, infraestrutura de sistemas, entre outros. Tudo isto serve como motivação para atualização e aprendizado de diversos tópicos relacionados ao desenvolvimento de \textit{softwares}.   

% Seção
\section{Objetivos}
A finalidade do projeto será demonstrar como pode ser feita a transferência de dados em tempo real entre servidores e clientes. Para isto, será feita a implementação de uma aplicação que irá realizar a comunicação por texto entre usuários, que serão localizados para comunicação através de localização geográfica em um mapa terrestre real.

O produto mínimo viável da aplicação será elaborado para um ambiente de testes local, não oferecendo um fluxo completo de funcionamento dos servidores em ambiente de produção, nem do fluxo de utilização de um usuário comum em sua interface. Abordaremos apenas alguns conceitos sobre escalabilidade e infraestrutura da aplicação em ambiente de produção, assim como apenas algumas funcionalidades necessárias para interação na interface gráfica.

Será explicado com mais clareza os problemas que acercam este projeto no tópico "Problematização", abordando especificamente quais são os ideais de implementação de uma solução em tempo real que tem características semelhante a desenvolvida neste trabalho, levando em conta aspectos como produto e serviço, já que ideologicamente um software é feito para realizar uma ou mais funções para um ou mais usuários, mas como citado, o foco é demonstrar fundamentalmente o funcionamento deste tipo de aplicação, e não oferecer um produto ou serviço. A problematização irá se perfazer em torno do estudo de uma forma de se implementar uma solução de tempo real para a \textit{web} envolvendo duas funcionalidades de representatividade para a circunstância.
