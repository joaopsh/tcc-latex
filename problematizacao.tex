\chapter{Problematização}
\section{Caracterização do Problema}
A internet oferece diversas possibilidades de integração e distribuição de dados em sua rede mundial. A necessidade cada vez maior de se coletar dados que são gerados a todo momento, cria a iniciativa de se laborar novas maneiras de envio e recepção de dados na rede.

A transmissão de dados em tempo real não é uma necessidade recente, desde que se começaram a desenvolver sistemas críticos de controle na \textit{web}, muitos destes já utilizavam desta forma de se transmitir dados para realizar suas tarefas. As técnicas e tecnologias utilizadas para alcançar tal feito eram diferentes e mais custosas, o que levava este tipo de característica da aplicação ser adotada apenas em casos específicos, onde deveria se contrabalancear com mais perícia a utilização.

Com os avanços das tecnologias de \textit{hardware}, \textit{software} e da internet como um todo, começaram a se criar novas possibilidades da implementação de \textit{features} tempo-real em aplicações cada vez mais próximas do usuário comum. Com tudo isso, podemos observar que muitas das aplicações de sucesso do mercado usufruem desta tecnologia, que permite uma conexão direta e persistente entre cliente e servidor, viabilizando que se possa receber e enviar dados a qualquer momento, atualizando em tempo real todos os interessados em uma determinada informação.

A possibilidade da informação instantânea tem muito a contribuir em diversas áreas, oferecendo vantagem competitiva para quem está no mundo dos negócios, antecipação de tomada de decisões, entre outros benefícios. Este mundo do tempo-real está a cada vez mais tomando conta do nosso presente como usuários comuns, e a tendência é que essa forma de compartilhamento instantâneo se torne mais natural.

Como as pessoas utilizam e disponibilizam dados, tem uma importância fundamental no impacto em que isso pode ter em suas vidas. Desde as necessidades no mundo dos negócios, às necessidades do cotidiano de cada um. Por parte de quem está inovando no setor, é preciso entender comportamentos e como não romper barreiras que irão fracassar a relação entre tecnologia e pessoas. 

A implementação de funcionalidades tempo-real definitivamente se tornou algo exequível para as aplicações atuais na \textit{web}, como \textit{push notifications} (notificações de eventos que são emitidas do servidor para o cliente), \textit{stats} (status de dados), interações em jogos de navegadores, entre outros. Os \textit{WebSockets} funcionam analogicamente como os \textit{sockets}, e nos permite fazer a comunicação entre processos através da rede. Como o próprio nome diz, os \textit{WebSockets} são especificamente feitos para este tipo de comunicação na \textit{web}, possibilitando uma conexão contínua entre códigos JavaScript executados nos \textit{browsers} e os servidores \text{web}. Estes serão base fundamental para a resolução do problema proposto neste trabalho, levando em conta que são a tecnologia mais atual.

\section{Solução Proposta}
Buscando explorar a usabilidade tempo-real na \textit{web}, iremos formular nossa solução no desenvolvimento de uma aplicação de comunicação por texto, também podendo ser chamada de \textit{chat}. Para que os usuários tenham conhecimento da existência um do outro na aplicação, existirá um mapa terrestre do mundo real que será marcado com pontos identificadores de cada usuário utilizando o sistema. Estes pontos terão como base a coordenada geográfica real do usuário que será fornecida quando o usuário entrar no sistema.

A escolha por uma aplicação deste modelo se deve a ideia de querer tornar o entendimento do emprego da tecnologia mais amigável para as pessoas que não possuem conhecimento técnico muito aprofundado. A troca mensagens explícita em uma conversa entre duas pessoas, essencialmente ilustra a ideia de transferência de dados em tempo real, e em conjunto com a localização geográfica, conseguimos dar um adendo de outra aplicação da tecnologia.

Iremos desenvolver uma interface com a pretensão de ser amigável ao usuário, utilizando conceitos de responsividade para a adaptação do conteúdo mostrado em telas de dispositivos diversos, inclusive móveis, tendo em vista que algumas estatísticas \cite{internet-traffic-stats1} apontam que dispositivos móveis já são responsáveis por pelo menos um quarto do todo o tráfego de dados gerado na internet, e com estimativas \cite{internet-traffic-stats2} \cite{internet-traffic-stats3} de aumento em cada ano. Mesmo não sendo o foco da aplicação ser um produto final, achamos que o aprendizado com este tipo de requisito seria conveniente. Em relação a organização da interface, a aplicação terá duas telas principais, sendo a primeira com as opções de cadastro através do fornecimento de alguns dados pessoais, e autenticação no sistema por \textit{e-mail} e senha. A segunda tela será o ambiente principal da aplicação, onde o usuário poderá navegar por um mapa terrestre em busca de marcações que identifiquem outro usuário, com o qual ele poderá interagir através de uma janela de \textit{chat} por texto que surge neste mesmo ambiente.

O principal componente da solução será o servidor da aplicação. Este será desenvolvido sobre a arquitetura REST (Representational State Transfer), que transmite e recebe dados da interface no formato JSON (JavaScript Object Notation). Funcionalidades como cadastro de usuário e autenticação farão uso deste modelo. A comunicação tempo-real será provida por um \textit{framework} que utiliza como base a tecnologia de WebSockets para conexões bidirecional de dados entre servidor e cliente. Apesar da tecnologia WebSocket ser a mais apropriada para a este tipo de conexão, o próprio \textit{framework} trata de usar outras técnicas de comunicação tempo-real, como \textit{long polling}, em casos de falta de compatibilidade dos navegadores com a tecnologia de WebSocket. Tudo isso acontece de forma transparente ao usuário e ao desenvolvedor, pois a utilização do \textit{framework} se dá através de API (Application Programming Interface) que oferece de forma simplificada todo o processo de conexão e comunicação.

A persistência dos dados será feita através de um banco de dados no NoSQL. A escolha por este tipo de banco de dados se deve as características da aplicação que exigem muitas operações de I/O (Input/Output) em dados com baixo nível de relacionamento com outros dados, onde este tipo de tecnologia NoSQL oferece melhor desempenho e praticidade.

A tecnologia utilizada como ambiente de execução no servidor, utiliza o JavaScript como linguagem padrão, além de oferecer características que também beneficiam o tipo de aplicação que iremos desenvolver, como \textit{single-theaded} (aplicação executada em apenas uma thread, processando um comando por vez), \textit{non-blocking IO} (operações de escrita e leitura de dados não bloqueiam a execução de código) e \textit{asynchronism} (não existe ordem de execução dos blocos de código). 

Considerando também alguns conceitos de infraestrutura, iremos adotar na arquitetura da aplicação algumas formas de permitir o escalonamento dos serviços através da distribuição dos servidores em diversas instâncias no mesmo servidor físico ou em outros de locais distintos. Para isso utilizaremos tecnologias para fazer a comunicação entre processos, com um banco de dados NoSQL centralizado e em memória, e interações \textit{stateless}, onde o servidor não guarda dados de sessão do usuário.