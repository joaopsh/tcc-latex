\chapter{Problematização}
\section{Caracterização do Problema}
A internet oferece diversas possibilidades de integração e distribuição de dados em sua rede mundial. A necessidade cada vez maior de se coletar dados que são gerados a todo momento, cria a iniciativa de se laborar novas maneiras de envio e recepção de dados na rede.

A transmissão de dados em tempo real não é uma necessidade recente, desde de que se começaram a desenvolver sistemas críticos de controle na \textit{web}, muitos destes já utilizavam desta forma de se transmitir dados para realizar suas tarefas. As técnicas e tecnologias utilizadas para alcançar tal feito eram diferentes e mais custosas, o que levava este tipo de característica da aplicação ser adotada apenas em casos específicos, onde deveria se contrabalancear com mais perícia a utilização.

Com os avanços das tecnologias de \textit{hardware}, \textit{software} e da internet como um todo, começaram a se criar novas possibilidades da implementação de \textit{features} tempo-real em aplicações cada vez mais próximas do usuário comum. Com tudo isso, podemos observar que muitas das aplicações de sucesso do mercado usufruem desta tecnologia, que permite uma conexão direta e persistente entre cliente e servidor, viabilizando que se possa receber e enviar dados a qualquer momento, atualizando em tempo real todos os interessados em uma determinada informação.

A implementação de funcionalidades tempo-real definitivamente se tornou algo exequível para as aplicações atuais na \textit{web}, como \textit{push notifications}, \textit{stats}, interações em jogos de navegadores, entre outros. A utilização de \textit{WebSockets}, que analogicamente funcionam como os \textit{sockets}, que fazem comunicação entre processos através da rede, os \textit{WebSockets} são especificamente feitos para este tipo de comunicação na \textit{web}, possibilitando uma conexão contínua entre códigos JavaScript executados nos \textit{browsers} e os servidores \text{web}. Estes serão base fundamental para a resolução do problema proposto neste trabalho, levando em conta que são a tecnologia mais atual.

\section{Solução Proposta}
Buscando explorar a usabilidade tempo-real na \textit{web}, iremos formular nossa solução no desenvolvimento de uma aplicação de comunicação por texto, também podendo ser chamado de \textit{chat}, onde para que os usuários tenham conhecimento da existência um do outro na aplicação, existirá um mapa terrestre do mundo real que será marcado com pontos identificadores de cada usuário utilizando o sistema. Estes pontos terão como base a coordenada geográfica real do usuário que será fornecida quando o usuário entrar no sistema.

A escolha por uma aplicação deste modelo se deve a ideia de querer tornar o entendimento do emprego da tecnologia mais amigável para as pessoas que não possuem conhecimento técnico muito aprofundado. A clareza que uma troca mensagens explícita em uma conversa entre duas pessoas, essencialmente ilustra a ideia de transferência de dados em tempo real. Em conjunto com a localização geográfica, conseguimos dar um adendo de outra aplicação da tecnologia.

Iremos desenvolver uma interface que com a pretensão de ser amigável ao usuário, utilizando conceitos de responsividade para a adaptação do conteúdo mostrado em telas de dispositivos diversos, inclusive móveis, tendo em vista que algumas estatísticas \cite{internet-traffic-stats1} apontam que dispositivos móveis já são responsáveis por pelo menos um quarto do todo o tráfego de dados gerado na internet, e com estimativas \cite{internet-traffic-stats2} \cite{internet-traffic-stats3} de aumento em cada ano. Mesmo não sendo o foco da aplicação ser um produto final, achamos que o aprendizado com este tipo de requisito seria conveniente. Em relação a organização da interface, a aplicação terá duas telas principais, sendo a primeira com as opções de cadastro através do fornecimento de alguns dados pessoais, e autenticação no sistema por \textit{e-mail} e senha. A segunda tela será o ambiente principal da aplicação, onde o usuário poderá navegar por um mapa terrestre em busca de marcações que identifiquem outro usuário, com o qual ele poderá interagir através de uma janela de \textit{chat} por texto que surge neste mesmo ambiente.

O servidor da aplicação será desenvolvido...

\section{Objetivos Específicos}